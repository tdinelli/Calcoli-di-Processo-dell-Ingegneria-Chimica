\documentclass[oneside]{article}

% ---------------------------------------------
% Importing packages
% ---------------------------------------------

% Encoding and font
\usepackage[utf8]{inputenc}
\usepackage{tgcursor}
\usepackage{hyperref}

% Different colors
\usepackage{xcolor}
\usepackage{color}
\definecolor{bluepoli}{cmyk}{0.4,0.1,0,0.4}

% Math
\usepackage{amsmath}
\usepackage{amsthm}

% Images
\usepackage{graphicx}
% \graphicspath{ {./Figures/} }

% Margins
\usepackage[a4paper, top=2cm, left=2.5cm, right=2.5cm, bottom=2cm]{geometry}

% Fancy header and footer
\usepackage{fancyhdr}
\pagestyle{fancy}
\fancyhf{}
\rhead{Calcoli di Processo dell'Ingegneria Chimica}
\lhead{Programming exercises}
\rfoot{Page \thepage}
\lfoot{Academic Year 2024-2025}


\usepackage{amsthm}
\usepackage{tcolorbox}
\tcbuselibrary{most}
\tcolorboxenvironment{proof}{% `proof' from `amsthm'
   blanker,
   breakable,
   left=5mm,
   before skip=10pt,
   after skip=10pt,
   borderline west={1mm}{0pt}{bluepoli}
}

% ---------------------------------------------
% Title
% ---------------------------------------------

\title{Additional programming exercises}
\author{Timoteo Dinelli\footnote{timoteo.dinelli@polimi.it}, Marco Mehl\footnote{marco.mehl@polimi.it}}
\date{3\textsuperscript{rd} of October 2024}

% ---------------------------------------------
% Begin of the document
% ---------------------------------------------

\begin{document}
\maketitle

\section*{INFO}
Solutions can be found on
\href{https://github.com/Titodinelli/Calcoli-di-Processo-dell-Ingegneria-Chimica}{GitHub}.

\section*{Exercises}

\begin{enumerate}
   \item Write a function that finds the maximum value and its position, in terms of row
      and column number, of the matrix M = magic(234) and compare the result obtained
      with the MATLAB builtin function max() and find().

   \item Write a MATLAB script that proves that the magic matrix definition is correct.
      And compare the result with a randomly generated one.

   \item Write a function that, taken as input an array A of n integers, returns its
      number of positive elements, without using predefined MATLAB library functions. For
      example:
      \begin{tcolorbox}[blanker, breakable, left=5mm, before skip=10pt, after skip=10pt,
         borderline west={1mm}{0pt}{bluepoli}]
         \textbf{Input}: A = [1, 5, -3, -9]; \\
         \textbf{Output}: ans = 2
      \end{tcolorbox}

   \item Write a MATLAB script that draws the plot of the function
      $f(x)=\frac{sin(x)}{x^4 + 2}$, with x within the interval $\left[-1, 1\right]$,
      with a spacing equal to $0.1$.

   \item Write a MATLAB program, that given two random points inside the Cartesian plane
      draws the corresponding line that pass through the two points:
      \begin{tcolorbox}[blanker, breakable, left=5mm, before skip=10pt, after skip=10pt,
         borderline west={1mm}{0pt}{bluepoli}]
         Remember that the slope and the intercept of a generic line $y=mx+q$ passing by
         two points, $P1(x_1, y_1),\:P2(x_2, y_2)$ are given by the following formulas:\\

         $m = \frac{y_2 - y_1}{x_2 - x_1}$ and $q = y_1 - mx_1$\\

         \textbf{Input}: $P_{1} = \left(4,-3\right)$; $P_{2} = \left(5,1\right)$;\\
         \textbf{Output}: $m = 4$, $q = -19$\\
      \end{tcolorbox}

   \item Given a square matrix \textbf{A}. We want to create a matrix \textbf{B} equal to
      the matrix \textbf{A} while replacing only the elements on the main diagonal with
      the average value of the corresponding rows.
      \begin{tcolorbox}[blanker, breakable, left=5mm, before skip=10pt, after skip=10pt,
         borderline west={1mm}{0pt}{bluepoli}]
         \textbf{Input}: \begin{equation*}
            A = \begin{bmatrix}
               1 & 2 & 3 \\
               4 & 5 & 6 \\
               7 & 8 & 9
            \end{bmatrix}
            \end{equation*}\\
         \textbf{Output}: \begin{equation*}
            B = \begin{bmatrix}
               2 & 2 & 3 \\
               4 & 5 & 6 \\
               7 & 8 & 8
            \end{bmatrix}
            \end{equation*}
      \end{tcolorbox}

   \item
      Given a square matrix A. You want to create a matrix B containing below the main
      diagonal all null elements, above the main diagonal all elements equal to the sum
      of all elements of matrix A, and on the main diagonal the corresponding elements of
      matrix A.
      \begin{tcolorbox}[blanker, breakable, left=5mm, before skip=10pt, after skip=10pt,
         borderline west={1mm}{0pt}{bluepoli}]
         \textbf{Input}: \begin{equation*}
            A = \begin{bmatrix}
               1 & 2 & 3 \\
               4 & 5 & 6 \\
               7 & 8 & 9
            \end{bmatrix}
            \end{equation*}\\
         \textbf{Output}: \begin{equation*}
            B = \begin{bmatrix}
               1 & 45 & 45 \\
               0 & 5 & 45 \\
               0 & 0 & 9
            \end{bmatrix}
            \end{equation*}
      \end{tcolorbox}

   \item Write a function that given a random vector returns the same vector but with the
      elements sorted in ascending order, by implementing a simple version of the {\it
      bubble sort} algorithm (\href{https://it.wikipedia.org/wiki/Bubble_sort}{reference}).
      \begin{tcolorbox}[blanker, breakable, left=5mm, before skip=10pt, after skip=10pt,
         borderline west={1mm}{0pt}{bluepoli}]
         \textbf{Pseudo code}:\\
         0. procedure BubbleSort(A:lista of elements to be sorted)\\
         1.\hspace{1em}change is true\\
         2.\hspace{1em}while scambio do\\
         3.\hspace{2em}change is false\\
         4.\hspace{2em}for i = 0 to length(A)-1  do\\
         5.\hspace{3em}if A[i] > A[i+1] then\\
         6.\hspace{4em}swap( A[i], A[i+1] ) \\
         7.\hspace{4em}change is true\\
         \textbf{Input}: v = [5, 4, 6, 8, 11];\\
         \textbf{Output}: ans = [4, 5, 6, 8, 11]
      \end{tcolorbox}
\end{enumerate}
\end{document}
