\documentclass[xcolor={dvipsnames,rgb}, aspectratio=169]{beamer}

%%% PACKAGES %%%
\usepackage[T1]{fontenc}
\usepackage{tgheros}

% Metropolis customization
\usetheme[sectionpage=none]{metropolis}
\setbeamercolor{background canvas}{bg=white}
\setbeamercolor{frametitle}{bg = white, fg=black}
\setbeamertemplate{sections/subsections in toc}[square]
\setbeamertemplate{footline}{
   \textcolor{bluepoli}{\rule{\paperwidth}{1pt}}
   \vskip4pt
   \hskip5pt \tiny Linear System Of Equations (Pt. 1) $|$ Calcoli di Processo dell' Ingegneria
   Chimica \hskip220pt \insertframenumber
   \vskip4pt
}

% color
\usepackage{color}
\usepackage{xcolor}
\usepackage{colortbl}
\definecolor{bluepoli}{cmyk}{0.4,0.1,0,0.4}
\definecolor{mygreen}{RGB}{1, 121,111}
\definecolor{myred}{RGB}{220, 20, 60}
\definecolor{mygreen}{RGB}{28,172,0}
\definecolor{mylilas}{RGB}{170,55,241}
\definecolor{codegreen}{rgb}{0,0.6,0}
\definecolor{codegray}{rgb}{0.5,0.5,0.5}
\definecolor{codepurple}{rgb}{0.58,0,0.82}
\definecolor{backcolour}{rgb}{0.95,0.95,0.92}
\definecolor{lightblue}{rgb}{56, 167, 232}

\colorlet{colorp}{NavyBlue}
\colorlet{colorT}{WildStrawberry}
\colorlet{colork}{OliveGreen}
\colorlet{colorM}{RoyalPurple}
\colorlet{colorNb}{Plum}
\colorlet{colorIs}{black}
\newcommand{\highlight}[2]{\colorbox{#1!17}{$#2$}}
\newcommand{\highlightdark}[2]{\colorbox{#1!47}{$#2$}}

% tikz
\usepackage{tikz}
\usetikzlibrary{positioning}
\usetikzlibrary{backgrounds}
\usetikzlibrary{arrows,shapes}
\usetikzlibrary{tikzmark}
\usetikzlibrary{calc}

% tcolorbox env
% Coloured box for styling theorems, proof, definitions
\usepackage[most]{tcolorbox}

\newtcolorbox{code}[2][]{
    enhanced jigsaw,
    colframe=bluepoli,
    interior hidden, 
    breakable,
    before skip=10pt,
    after skip=10pt
}

% URL and Hyperref
\usepackage{hyperref}
\hypersetup{
    colorlinks=true,
    linkcolor=blue,
    filecolor=magenta,
    urlcolor=blue,
    pdftitle={Overleaf Example},
    pdfpagemode=FullScreen,
}
\usepackage{url}

% Math stuff
\usepackage{amsmath}
\usepackage{amssymb}
\usepackage{mathtools}
\usepackage{blkarray}
\usepackage{multirow}

% Wrapfig
\usepackage{wrapfig}

% Bibliography
\usepackage[
backend=biber,
style=alphabetic,
sorting=ynt
]{biblatex}
\addbibresource{bibliography.bib}

%%% TITLE %%%
\title{Linear System Of Equations.\\Part 1}
\subtitle{Calcoli di Processo dell' Ingegneria Chimica}
\author[Dinelli, Mehl]{\textbf{Timoteo~Dinelli}, \textbf{Marco~Mehl}}
\institute{
   \inst{} Department of Chemistry, Materials and Chemical Enginering, G. Natta.
   Politecnico di Milano.\\
   email: timoteo.dinelli@polimi.it \\
   email: marco.mehl@polimi.it \\
}
\date{18\textsuperscript{th} of October 2024.}

\begin{document}
% external files inclusion
% Double underline
\def\doubleunderline#1{\underline{\underline{#1}}}

% \newcommand{\zm}{%
%    \begin{bmatrix}
%       X_{11} & X_{12} & \cdots & X_{1p} \\
%       X_{12} & X_{22} & \cdots & X_{2p} \\
%       \vdots & \vdots & \tikzmarknode{Is}{\highlight{colorT}{X_{ij}}} & \vdots \\
%       X_{n1} & X_{n2} & \cdots & X_{np} \\
%    \end{bmatrix}%
% }

\makeatletter
\newcommand{\DrawLine}{%
  \begin{tikzpicture}
  \path[use as bounding box] (0,0) -- (\linewidth,0);
  \draw[color=bluepoli,dashed,dash phase=2pt]
        (0-\kvtcb@leftlower-\kvtcb@boxsep,0)--
        (\linewidth+\kvtcb@rightlower+\kvtcb@boxsep,0);
  \end{tikzpicture}%
  }
\makeatother

{%
   \setbeamertemplate{footline}{}
   \begin{frame}{}
      \maketitle
      \begin{tikzpicture}[overlay, remember picture]
         \node[above left=6.5cm and .01cm of current page.south east] {
         \includegraphics[trim=1cm 1cm 1.5cm 1cm, clip=true, width=6cm]{
            ./../../Introduction to Matlab/slides/figures/_static/ING_IND_INF-eps-converted-to.pdf
         }
      };
      \end{tikzpicture}
   \end{frame}
}

\begin{frame}{Linear System Of Equations}
   \small{In mathematics, and particularly in linear algebra, a system of linear
   equations, also called a linear system, is a system composed of several \alert{linear
   equations} that must all be verified simultaneously. A solution of the system is a
   vector whose elements are the solutions of the equations that make up the system, that
   is, such that when substituted for the unknowns make the equations identities.}

   From the classical representation to the \alert{\textbf{matricial}} form:
   \begin{columns}
   \column{0.5\textwidth}
      \begin{equation*}
      \begin{cases}
         a_{1,1}x_{1} + a_{1,2}x_{2} + \ldots + a_{1,n}x_{n}\quad = \quad b_{1} \\
         a_{2,1}x_{1} + a_{2,2}x_{2} + \ldots + a_{2,n}x_{n}\quad = \quad b_{2} \\
         \quad \quad \vdots                                                     \\
         a_{n,1}x_{1} + a_{n,2}x_{2} + \ldots + a_{n,n}x_{n}\quad = \quad b_{n}
      \end{cases}
      \end{equation*}
   \column{0.5\textwidth}
      \begin{equation*}
         \tikzmarknode{Amatrix}{\highlight{colork}{
            \begin{bmatrix}
               a_{1,1} & a_{1,2} & \ldots & a_{1,n} \\
               a_{2,1} & a_{2,2} & \ldots & a_{2,n} \\
               \vdots  & \vdots  & \ddots & \vdots  \\
               a_{n,1} & a_{n,2} & \ldots & a_{n,n}
            \end{bmatrix}
         }}\:
         \tikzmarknode{xvector}{\highlight{colorp}{
            \begin{bmatrix}
               x_{1}  \\
               x_{2}  \\
               \vdots \\
               x_{n}
            \end{bmatrix}
         }}
         =
         \tikzmarknode{xvector}{\highlight{colorT}{
            \begin{bmatrix}
               b_{1}  \\
               b_{2}  \\
               \vdots \\
               b_{n}
            \end{bmatrix}
         }}
      \end{equation*}
   \end{columns}

   \begin{equation*}
      \textcolor{colork}{\textbf{\underline{\underline{A}}}} \:
      \textcolor{colorp}{\textbf{\underline{x}}} =
      \textcolor{colorT}{\textbf{\underline{b}}}
      \quad \longrightarrow \quad
      \textbf{\underline{x}} = \textbf{\underline{\underline{A}}}^{\textbf{-1}} \:
      \textbf{\underline{b}}
   \end{equation*}
\end{frame}

\begin{frame}{}
    \begin{equation*}
        \textbf{\underline{x}} = \textbf{\underline{\underline{A}}}^{\textbf{-1}} \:
        \textbf{\underline{b}}
    \end{equation*}

    This solution is impracticable (since requires the exact inversion of
    $\textbf{\underline{\underline{A}}}$)! So two types of method have been invented, the
    so called \textbf{DIRECT} methods and the \textbf{ITERATIVE} methods. \\ Now let's
    consider a $3 \times 3$ system of equations like the one below: \vspace{1cm}

    \begin{columns}
    \column{0.5\textwidth}
    \begin{equation*}
        \begin{cases}
            3x + 89y + 66z = 87 \\
            65y + 9z = 7 \\
            46z = 3 \\
        \end{cases}
    \end{equation*}
    \column{0.5\textwidth}
    \begin{equation*}
        \begin{bmatrix}
        3 & 89 & 66 \\
        0 & 65 & 9 \\
        0 & 0 & 46
        \end{bmatrix}
        \begin{bmatrix}
        x \\ y \\ z
        \end{bmatrix} = 
        \begin{bmatrix}
        87 \\ 7 \\ 3
        \end{bmatrix}
    \end{equation*}
    \end{columns}

    So the goal is to get $\textbf{\underline{\underline{A}}}$ and after some operations
    get its upper triangular form.
\end{frame}

\begin{frame}{Gauss elimination}
\vspace{-.5cm}
Given a linear system of equations in the form $\textbf{A}\textbf{x} = \textbf{b}$, it is
convenient to introduce the augmented matrix
$\textbf{A}^{\textbf{*}} = \left[\textbf{A}\:|\:\textbf{b}\right]$
\vspace{-.2cm}
\begin{equation*}
\left[\textbf{A}\:\vline \:\textbf{b}\right] = 
\begin{bmatrix}
   a_{1,1}^{(0)} & \ldots & a_{1,n}^{(0)} & \vline & b_{1}^{(0)} \\
   \vdots        & \ddots & \vdots        & \vline & \vdots \\
   a_{n,1}^{(0)} & \ldots & a_{n,n}^{(0)} & \vline & b_{n}^{(0)} \\
\end{bmatrix}
\end{equation*}

After $n-1$ steps we obtain the following system:

\begin{equation*}
\left[\textbf{A} \: \vline \: \textbf{b}\right] = 
\begin{bmatrix}
   a_{1,1}^{(0)} & \ldots        & \ldots & a_{1,n}^{(0)}   & \vline & b_{1}^{(0)}   \\
   0             & a_{2,2}^{(1)} & \ldots & a_{2,n}^{(1)}   & \vline & b_{2}^{(1)}   \\
   \vdots        & \ddots        & \ldots & \ldots          & \vline & \vdots        \\
   0             & \ldots        & 0      & a_{n,n}^{(n-1)} & \vline & b_{n}^{(n-1)} \\
\end{bmatrix}
\end{equation*}
\end{frame}

\begin{frame}{Key IDEA!}

   \textcolor{colorT}{Let's use lower triangular matrices}.

   \begin{equation*}
       \begin{bmatrix}
       a_{1,1} & a_{1,2} & \ldots & a_{1,n-1}  & a_{1,n} \\
       \tikzmarknode{zeroes}{\highlight{colorT}{0}}       & a_{2,2} & \ldots & a_{2, n-1} & a_{2,n} \\
       \tikzmarknode{zeroes}{\highlight{colorT}{0}}       & \tikzmarknode{zeroes}{\highlight{colorT}{0}}       & \ldots & a_{3, n-1} & a_{3,n} \\
       \tikzmarknode{zeroes}{\highlight{colorT}{\vdots}}  & \tikzmarknode{zeroes}{\highlight{colorT}{\vdots}}  &
          \tikzmarknode{zeroes}{\highlight{colorT}{\ddots}} & \vdots     & \vdots  \\
       \tikzmarknode{zeroes}{\highlight{colorT}{0}}       &
          \tikzmarknode{zeroes}{\highlight{colorT}{0}}       &
          \tikzmarknode{zeroes}{\highlight{colorT}{\ldots}} &
          \tikzmarknode{zeroes}{\highlight{colorT}{0}}          & a_{n,n}
       \end{bmatrix} \quad \textbf{x} = \textbf{b}^{*}
   \end{equation*}
\end{frame}

\begin{frame}{LU factorization/decomposition}
    \begin{equation*}
        \textbf{\underline{\underline{A}}} \: \textbf{\underline{x}} =
        \textbf{\underline{b}}
    \end{equation*}
    \begin{equation*}
        \textbf{\underline{\underline{A}}} \: \textbf{\underline{x}} \: \: = \: \:
        \textbf{\underline{\underline{L}}} \textbf{\underline{\underline{U}}} \: \:
        \textbf{\underline{x}} \: \: = \: \: \textbf{\underline{b}}
    \end{equation*}
    \begin{equation*}
        \textbf{\underline{\underline{U}}} \: \textbf{\underline{x}} =
        \textbf{\underline{y}}  \longrightarrow \textbf{\underline{\underline{L}}} \quad
        \textbf{\underline{y}} =  \textbf{\underline{b}}
    \end{equation*}
    \quad \quad \quad \quad \quad \quad \quad \quad \quad \quad Solution:
    \begin{equation*}
        \textbf{\underline{\underline{L}}} \: \textbf{\underline{y}} =
        \textbf{\underline{b}} \longrightarrow \textbf{\underline{\underline{U}}} \:
        \textbf{\underline{x}} = \textbf{\underline{y}}
    \end{equation*}
    \begin{columns}
    \column{0.5\textwidth}
    \begin{equation*}
        \textbf{L} = \begin{bmatrix}
        1 & 0 & 0 \\
        c_{2,1} & 1 & 0 \\
        c_{3,1} & c_{3,2} & 1
        \end{bmatrix}
    \end{equation*}
    \column{0.5\textwidth}
    \begin{equation*}
        \textbf{U} = \begin{bmatrix}
        d_{1,1} & d_{1,2} & d_{1,3} \\
        0 & d_{2,2} & d_{2,3} \\
        0 & 0 & d_{3,3}
        \end{bmatrix}
    \end{equation*}
    \end{columns}
\end{frame}

\begin{frame}{WHY LU?}
\begin{itemize}
    \item[$\blacktriangleright$] Once \textbf{\underline{\underline{A}}} has been
       decomposed, it is possible to solve multiple problems having different \textbf{b}
       values (useful for other numerical methods).

    \item[$\blacktriangleright$] Once \textbf{\underline{\underline{A}}} has been
       decomposed, it is possible to solve: $A^{T}x = c$ without doing a factorization
       for $\textbf{\underline{\underline{A}}}^{T}$.

    \item[$\blacktriangleright$] The matrices \textbf{\underline{\underline{L}}} and
       \textbf{\underline{\underline{U}}} can be obtained using different methods than
       the Gauss elimination. Pay attention to the built-in function \textcolor{blue}{lu}
       of MATLAB (read the documentation).

    \item[$\blacktriangleright$] If \textbf{\underline{\underline{A}}} is modified,
       techniques that update \textbf{\underline{\underline{L}}} e
       \textbf{\underline{\underline{U}}} accordingly to generate the new matrix
       $\textbf{\underline{\underline{A}}}^{*}$.
\end{itemize}
\end{frame}

{%
\setbeamertemplate{footline}{}
\begin{frame}[standout]
	Exercises
\end{frame}
}

\begin{frame}{Triangularize U}
   Write a function that takes the matrix A of size ($n\times n$) and b ($n \times 1$) as
   the input and outputs a triangular upper matrix using the the Gauss elimination method
   (without pivoting/balancing). \\
   The function: \textcolor{blue}{function}[A,b]=triangularize\_U(A,b) Will generate the
   augmented matrix $\textbf{\underline{\underline{W}}}$ and, using two nested
   \textit{for} loops, the first with index \textit{i} ranging from \textit{1} to
   \textit{n-1}, the second (index \textit{j}) from \textit{i+1} to \textit{n}, that will
   operate the linear combination that eliminates the element \textit{(j,i)} from
   $\textbf{\underline{\underline{W}}}$. A second function
   \textcolor{blue}{solve\_upper\_triangular\_system(A,b)} will take the output of the
   first function and will solve the system using the algorithms discussed in class
   (solution either by row or by column). Finally, write a function
   \textcolor{blue}{my\_linear\_solver(A,b)} that calls the first two and returns the
   solution of a linear system. And compare your solution with the one obtained using the
   built-in function \textcolor{blue}{linsolve} or \textcolor{blue}{mldivide}.
\end{frame}

\begin{frame}{}
To test out if your algorithm works properly use the following systems of equations:
\vspace{1.5cm}
\begin{columns}
\column{0.5\textwidth}
   \begin{equation*}
       \begin{cases}
       x + 2y - z + 2t = 3 \\
       x + 2z + t = 1 \\
       2x + y - 2t = 1 \\
       -z + t = 2
       \end{cases}
   \end{equation*}
\column{0.5\textwidth}
   \begin{equation*}
       \begin{cases}
       x + 45y - t = 6 \\
       x - 3t = 12 \\
       x + y + z + 6 = 0 \\
       x - y + z + t = 12
       \end{cases}
   \end{equation*}
\end{columns}
\end{frame}

\begin{frame}{LU}
   Let's modify the previously implemented code and implemented a function to perform the
   \alert{LU} decomposition of a given square matrix. Then combine this function inside
   our linear solver and compare the results.
\end{frame}
%%%%%%%%%%%%%%% CLOSING
{%
\setbeamertemplate{footline}{}
\begin{frame}[standout]
	Thank you for the attention!
\end{frame}
}

\end{document}
