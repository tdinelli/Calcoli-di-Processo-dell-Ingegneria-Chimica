\documentclass[xcolor={dvipsnames,rgb}, aspectratio=169]{beamer}

%%% PACKAGES %%%
\usepackage[T1]{fontenc}
\usepackage{tgheros}

% Metropolis customization
\usetheme[sectionpage=none]{metropolis}
\setbeamercolor{background canvas}{bg=white}
\setbeamercolor{frametitle}{bg = white, fg=black}
\setbeamertemplate{sections/subsections in toc}[square]
\setbeamertemplate{footline}{
   \textcolor{bluepoli}{\rule{\paperwidth}{1pt}}
   \vskip4pt
   \hskip5pt \tiny Introduction to Matlab $|$ Calcoli di Processo dell' Ingegneria
   Chimica \hskip250pt \insertframenumber
   \vskip4pt
}

% color
\usepackage{color}
\usepackage{xcolor}
\usepackage{colortbl}
\definecolor{bluepoli}{cmyk}{0.4,0.1,0,0.4}
\definecolor{mygreen}{RGB}{1, 121,111}
\definecolor{myred}{RGB}{220, 20, 60}
\definecolor{mygreen}{RGB}{28,172,0}
\definecolor{mylilas}{RGB}{170,55,241}
\definecolor{codegreen}{rgb}{0,0.6,0}
\definecolor{codegray}{rgb}{0.5,0.5,0.5}
\definecolor{codepurple}{rgb}{0.58,0,0.82}
\definecolor{backcolour}{rgb}{0.95,0.95,0.92}
\definecolor{lightblue}{rgb}{56, 167, 232}

\colorlet{colorp}{NavyBlue}
\colorlet{colorT}{WildStrawberry}
\colorlet{colork}{OliveGreen}
\colorlet{colorM}{RoyalPurple}
\colorlet{colorNb}{Plum}
\colorlet{colorIs}{black}
\newcommand{\highlight}[2]{\colorbox{#1!17}{$#2$}}
\newcommand{\highlightdark}[2]{\colorbox{#1!47}{$#2$}}

% tikz
\usepackage{tikz}
\usetikzlibrary{positioning}
\usetikzlibrary{backgrounds}
\usetikzlibrary{arrows,shapes}
\usetikzlibrary{tikzmark}
\usetikzlibrary{calc}

% tcolorbox env
% Coloured box for styling theorems, proof, definitions
\usepackage[most]{tcolorbox}

\newtcolorbox{code}[2][]{
    enhanced jigsaw,
    colframe=bluepoli,
    interior hidden, 
    breakable,
    before skip=10pt,
    after skip=10pt
}

% URL and Hyperref
\usepackage{hyperref}
\hypersetup{
    colorlinks=true,
    linkcolor=blue,
    filecolor=magenta,      
    urlcolor=blue,
    pdftitle={Overleaf Example},
    pdfpagemode=FullScreen,
}
\usepackage{url}

% Math stuff
\usepackage{amsmath}
\usepackage{amssymb}
\usepackage{mathtools}
\usepackage{blkarray}
\usepackage{multirow}

% Wrapfig
\usepackage{wrapfig}

% Bibliography
\usepackage[
backend=biber,
style=alphabetic,
sorting=ynt
]{biblatex}
\addbibresource{bibliography.bib}

%%% TITLE %%%
\title{Numbers, errors and computers.}
\subtitle{Calcoli di Processo dell' Ingegneria Chimica}
\author[Dinelli, Mehl]{\textbf{Timoteo~Dinelli}, \textbf{Marco~Mehl}}
\institute{
   \inst{} Department of Chemistry, Materials and Chemical Enginering, G. Natta.
   Politecnico di Milano.\\
   email: timoteo.dinelli@polimi.it \\
   email: marco.mehl@polimi.it \\
}
\date{4\textsuperscript{th} of October 2024.}

\begin{document}
% external files inclusion
% Double underline
\def\doubleunderline#1{\underline{\underline{#1}}}

% \newcommand{\zm}{%
%    \begin{bmatrix}
%       X_{11} & X_{12} & \cdots & X_{1p} \\
%       X_{12} & X_{22} & \cdots & X_{2p} \\
%       \vdots & \vdots & \tikzmarknode{Is}{\highlight{colorT}{X_{ij}}} & \vdots \\
%       X_{n1} & X_{n2} & \cdots & X_{np} \\
%    \end{bmatrix}%
% }

\makeatletter
\newcommand{\DrawLine}{%
  \begin{tikzpicture}
  \path[use as bounding box] (0,0) -- (\linewidth,0);
  \draw[color=bluepoli,dashed,dash phase=2pt]
        (0-\kvtcb@leftlower-\kvtcb@boxsep,0)--
        (\linewidth+\kvtcb@rightlower+\kvtcb@boxsep,0);
  \end{tikzpicture}%
  }
\makeatother

{%
   \setbeamertemplate{footline}{}
   \begin{frame}{}
      \maketitle
      \begin{tikzpicture}[overlay, remember picture]
         \node[above left=6.5cm and .01cm of current page.south east] {
            \includegraphics[trim=1cm 1cm 1.5cm 1cm, clip=true, width=6cm]{
               ./../../Introduction to Matlab/slides/figures/_static/ING_IND_INF-eps-converted-to.pdf
            }
         };
      \end{tikzpicture}
   \end{frame}
}

\begin{frame}{Numbers representation.}
   ``{\it In computing, \alert{floating-point arithmetic (\textbf{FP})} is arithmetic
   that represents subsets of real numbers using an integer with a fixed precision,
   called the significand, scaled by an integer exponent of a fixed base. Numbers of this
   form are called floating-point numbers. For example, 12.345 is a floating-point number
   in base ten with five digits of precision.}''
   \href{https://en.wikipedia.org/wiki/Floating-point_arithmetic}{Wikipedia}.

   \begin{equation*}
      12.345 = \tikzmarknode{mantissa}{\highlight{colork}{12345}} \: \times \:
      \tikzmarknode{exponent}{\highlight{colorp}{10^{-3}}}
   \end{equation*}

   \begin{tikzpicture}[overlay, remember picture, >=stealth, nodes={align=left,inner ysep=1pt},<-]
      \node[anchor=west, color=colork!85, yshift=-1em, xshift=-5.2em] (mantissa_text) at
      (mantissa.south)
      {\textsf{\footnotesize Mantissa}};
      \draw [color=colork](mantissa.south) |- ([yshift=-0.2ex]mantissa_text.south west);

      \node[anchor=west, color=colorp!85, yshift=-1em, xshift=3.2em] (exp_text) at (exponent.south)
      {\textsf{\footnotesize Exponent}};
      \draw [color=colorp](exponent.south) |- ([yshift=-0.2ex]exp_text.south east);
   \end{tikzpicture}

   Additional RECOMMENDED read:
   \begin{itemize}
      \item[$\blacktriangleright$]
         \href{https://en.wikibooks.org/wiki/A-level_Computing/AQA/Paper_2/Fundamentals_of_data_representation/Floating_point_numbers}{Floating
         point representations}.
      \item[$\blacktriangleright$] \href{https://medium.com/@lmpo/understanding-model-quantization-for-llms-1573490d44ad}{LLM quantization}.
   \end{itemize}
\end{frame}


\begin{frame}{Single or Double precision?}
    In any computer, the difference between single and double precision reflects on the
    number of bits used to represent a real number.
    \begin{itemize}
        \item[$\blacktriangleright$] Single precision floating point number 32 bits (4
           bytes). Safe use with $\sim 7/8$ decimal digits.
        \item[$\blacktriangleright$] Double precision floating point number 64 bits (8
           bytes). Safe use with $\sim 15/16$ decimal digits.
    \end{itemize}
\end{frame}

\begin{frame}{Elementary operations}
\vspace{-.5cm}
   \small{Initialize a single precision variable in MATLAB (e.g. x = $1e^{+25}$) using
   the function:\\
   x = \alert{single}(1.e+25).

   Working in single precision predict and calculate from x = 1.e+25  and y = 1.e+18 the
   following values of z:}

   \begin{table}
   \centering
   \begin{tabular}{l|l|l} 
   \hline
   \multicolumn{1}{l}{} & \multicolumn{1}{l}{\textbf{Single Precision}} & \textbf{Double Precision}  \\ 
   \hline
      $z = x * y$ & inf & 1.0000e+43 \\
      \hline
      $z = x / y$ & 10000000 & 10000000 \\
      \hline
      $z = y / x$ & 1.0000e-07 & 1.0000e-07 \\
      \hline
      $z = x^2$ & inf & 1.0000e+50 \\
      \hline
      $z = y^2$ & 1.0000e+36 & 1.0000e+36 \\
      \hline
      $z = 1./(x*y)$ & 0 & 1.0000e-43 \\
      \hline
      $z = 1./x/y$ & 9.9492e-44 & 1.0000e-43 \\
      \hline
      $z = y + 1e10$ & 1.0000e+18 & 1.0000e+18 \\
      \hline
      $z = x * y / (x * y + 1)$ & NaN & 1 \\
   \hline
   \end{tabular}
   \end{table}
\end{frame}

{%
   \setbeamertemplate{footline}{}
   \begin{frame}[standout]
      Exercises
   \end{frame}
}

\begin{frame}{Compute the MACHEPS}
   Epsilon, or machine epsilon, is an important number in computing. Machine epsilon
   gives the distance between a number and the next largest floating point number on your
   computer. This is important to calculate, as the size of the floating point number may
   lead to round-off errors for certain calculations. Calculating machine epsilon can be
   done a number of ways, and many programming languages have built-in functions that can
   determine this value. However, it also can be determined algorithmically with a fairly
   simple routine. Write a function to determine the macheps of a generic number. Plot
   the results for numbers from ranging from 0 to 10.

   \begin{itemize}
      \item[$\blacktriangleright$] The strategy here is to iterate as long as the
         difference between $n$ and $n + epsilon / 2 > 0$ by halving at each iteration
         the value of the machine epsilon until convergence.
   \end{itemize}
\end{frame}

\begin{frame}{Sum of the inverse of numbers}
   Write a script which calculates:
   \begin{equation*} \sum_{n = 1}^{1000000}\frac{1}{n} \end{equation*}

   in \alert{single} and \alert{double} precision. Then compare with the results obtained
   inverting the order of the sum, so by computing:

   \begin{equation*} \sum_{n=1000000}^{1}\frac{1}{n} \end{equation*}
\end{frame}

\begin{frame}{The babilonian method}
\textcolor{mylilas}{Write a function that implements the Babilonian method to compute the
   square root of a number with a precision of four decimal figures}.
\begin{enumerate}
   \item \textbf{Make an Initial guess}.
      Guess any positive number $x_{0}$.
   \item \textbf{Improve the first guess}.
      Apply the formula $x_{1} = \frac{x_{0} + \frac{S}{x_{0}}}{2}$. The number $x_{1}$
      is a better approximation to $\sqrt{S}$.
   \item \textbf{Iterate until convergence}.
      Apply the formula $x_{n+1} = \frac{x_{n} + \frac{S}{x_{n}}}{2}$ until the
      convergence is reached.
\end{enumerate}
\textbf{Convergence} is reached when the digits of $x_{n+1}$ and $x_{n}$ agree to as many
   decimal places as you desire.
\end{frame}
% \begin{frame}{}
% \begin{tcolorbox}[colback=white,colframe=bluepoli]
%    $>>$ \textcolor{blue}{function} SquareRoot = ComputeSquareRoot(S)\\
%    $>>$ \hspace{1em}iter = 0;\\
%    $>>$ \hspace{1em}x0 = S;\\
%    $>>$ \hspace{1em}y = 0.5*(x0+S/x0);\\
%    $>>$ \hspace{1em}\textcolor{blue}{while} abs(x0-y)>1e-5 \&\& iter<50\\
%    $>>$ \hspace{1em}\hspace{1em}x0 = y;\\
%    $>>$ \hspace{1em}\hspace{1em}y = 0.5*(x0+S/x0);\\
%    $>>$ \hspace{1em}\hspace{1em}iter = iter+1;\\
%    $>>$ \hspace{1em}\textcolor{blue}{end}\\
%    $>>$ \hspace{1em}disp(['Number of iteration to reach convergence: ', num2str(iter)])\\
%    $>>$ \hspace{1em}SquareRoot = y;\\
%    $>>$ \textcolor{blue}{end}\\
% \end{tcolorbox}
% \end{frame}

\begin{frame}{Vancouver: a nickel at a time}
   \vspace{-.5cm}
   \small{Analogously to what happened on the Vancouver stock market
   (\href{https://slate.com/technology/2019/10/round-floor-software-errors-stock-market-battlefield.html}{reference}),
   starting from a stock value of 1000, check what happens when a random variation of
   $\pm1\%$ in its value is iterated for 10000 times. Try to round or approximate
   (truncate) to the lower or upper \alert{second} decimal figure. Compare the results
   with the number obtained using the computer precision.

   Please verify on the help page in Matlab how the functions {\it rand, floor, ceil,
   fix, round} work.

   % \begin{tcolorbox}[colback=white,colframe=bluepoli]
   \begin{tcolorbox}[colback=white,colframe=bluepoli, sidebyside]
      $>>$ $a = 1.2345$\\
      $>>$ $ceil(a*100)/100 \longrightarrow ans = 1.2400$\\
      $>>$ $floor(a*100)/100 \longrightarrow ans = 1.2300$\\
      $>>$ $round(a*100)/100 \longrightarrow ans = 1.2300$
      \tcblower
      $>>$ $a = -1.2345$\\
      $>>$ $ceil(a*100)/100 \longrightarrow ans = -1.2300$\\
      $>>$ $floor(a*100)/100 \longrightarrow ans = -1.2400$\\
      $>>$ $round(a*100)/100 \longrightarrow ans = -1.2300$
   \end{tcolorbox}}
\end{frame}

%%%%%%%%%%%%%%% CLOSING
{%
   \setbeamertemplate{footline}{}
   \begin{frame}[standout]
      Thank you for the attention!
   \end{frame}
}

% \appendix % This to turnoff numbering and stuff
% \begin{frame}{References}
%    % \bibliographystyle{alpha}
%    \printbibliography
% \end{frame}
% \begin{tcolorbox}[colback=white,colframe=bluepoli]
%    $>>$
%    $>>$
%    $>>$
%    $>>$
%    $>>$
%    $>>$
%    $>>$
%    $>>$
%    $>>$
%    $>>$
%    $>>$
%    $>>$
% \end{tcolorbox}
\end{document}
