\documentclass[oneside]{article}

% ---------------------------------------------
% Importing packages
% ---------------------------------------------

% Encoding and font
\usepackage[utf8]{inputenc}
\usepackage{tgcursor}
\usepackage{hyperref}
\usepackage{wrapfig}

% Different colors
\usepackage{xcolor}
\usepackage{color}
\definecolor{bluepoli}{cmyk}{0.4,0.1,0,0.4}

% Math
\usepackage{amsmath}
\usepackage{amsthm}

% Chemistry
\usepackage[version=4]{mhchem}

% Images
\usepackage{graphicx}
% \graphicspath{ {./Figures/} }

% Margins
\usepackage[a4paper, top=2cm, left=2.5cm, right=2.5cm, bottom=2cm]{geometry}

% Fancy header and footer
\usepackage{fancyhdr}
\pagestyle{fancy}
\fancyhf{}
\rhead{Calcoli di Processo dell'Ingegneria Chimica}
\lhead{Additional Random Exercises}
\rfoot{Page \thepage}
\lfoot{Academic Year 2024-2025}


\usepackage{amsthm}
\usepackage{tcolorbox}
\tcbuselibrary{most}
\tcolorboxenvironment{proof}{% `proof' from `amsthm'
   blanker,
   breakable,
   left=5mm,
   before skip=10pt,
   after skip=10pt,
   borderline west={1mm}{0pt}{bluepoli}
}
\usepackage{tikz}
\usetikzlibrary{arrows.meta}

% ---------------------------------------------
% Title
% ---------------------------------------------

\title{Additional (Random) Exercises}
\author{Timoteo Dinelli\footnote{timoteo.dinelli@polimi.it}, Marco Mehl\footnote{marco.mehl@polimi.it}}
% \date{13\textsuperscript{th} of December 2024}

% ---------------------------------------------
% Begin of the document
% ---------------------------------------------

\begin{document}
\maketitle

\section*{Exercise 1}
Inside a tank in a chemical plant, there is a mixture composed of hydrogen, ammonia, nitrogen, and water. Inside the tank, the pressure is $0.15$ [bar] and $25$ [$^{\circ}$C]. The components are distributed in two different phases: liquid and vapor. Initially, there are 3 moles of hydrogen, 1 of nitrogen, and 5 of water. Calculate the molar fractions of the components in both phases making the following assumptions:

\begin{itemize}
   \item The gas phase is treated as a perfect gas
   \item The liquid mixture is ideal
\end{itemize}

The following table shows the Gibbs free energy of formation values in the pure perfect gas state at $298.15$ [K] and $0.15$ [bar]. Additionally, the Henry's constant values and Vapor Pressure at $298.15$ [K] are reported.

\begin{table}[htp]
\begin{tabular}{l|l|l|l}
\textbf{Compound} & $\Delta g^{\circ}_{f,j}$ [kJ$\cdot$mol$^{-1}$] & \textbf{Henry's Constant [Pa] $H_{j}$} & \textbf{Vapor Pressure [Pa] $(P^{\circ}_{j})$} \\
\hline
$H_{2}$       & 0                   & $7.158 \cdot 10^{10}$                &                             \\
$N_{2}$       & 0                   & $9.244 \cdot 10^{10}$                 &                             \\
$NH_{3}$      & -16.33              & $9.758 \cdot 10^{4}$                  &                             \\
$H_{2}O$      &                     &                            & $3.167 \cdot 10^{3}$                  
\end{tabular}
\end{table}

In general, using the extent of reaction method, we can write the following system, which relates the extent of reaction ($\lambda$) with the moles of species in each phase. (Note: superscript V indicates species j in Vapor phase while superscript L indicates species j in liquid phase)

\begin{equation}
   \begin{cases}
       n_{N_{2}^{V}} = 1 - 0.5\lambda_{1} - \lambda_{2} \\
       n_{H_{2}^{V}} = 3 - 1.5\lambda_{1} - \lambda_{3} \\
       n_{NH_{3}^{V}} = \lambda_{1} - \lambda_{4} \\
       n_{H_{2}O^{V}} = 5 - \lambda_{5} \\
       n_{N_{2}^{L}} = \lambda_{2} \\
       n_{H_{2}^{L}} = \lambda_{3} \\
       n_{NH_{3}^{L}} = \lambda_{4} \\
       n_{H_{2}O^{L}} = \lambda_{5} 
   \end{cases}
\end{equation}

Additionally, the following relations hold, where $n_{T}^{L}$ is the total moles present in the liquid phase, $n_{T}^{V}$ is the total moles present in vapor phase, $x_{i}$ and $y_{i}$ are the molar fractions of species i in liquid and vapor phase respectively:

\begin{equation}
   n_{T}^{L} = \lambda_{2} + \lambda_{3} + \lambda_{4} + \lambda_{5} 
\end{equation}

\begin{equation}
   n_{T}^{V} = 9 - \lambda_{1} - \lambda_{2} - \lambda_{3} - \lambda_{4} - \lambda_{5}
\end{equation}

\begin{equation}
   x_{i} = \frac{n_{i}^{L}}{n_{T}^{L}}
\end{equation}

\begin{equation}
   y_{i} = \frac{n_{i}^{V}}{n_{T}^{V}}
\end{equation}

Finally, using pure perfect gas at $298.15$ [K] and $0.15$ [bar] as reference, assuming perfect gas and ideal liquid mixture, the system that solves our problem takes the following form:

\begin{equation}
   \begin{cases}
       \frac{y_{NH_{3}^{V}}}{y_{N_{2}}^{0.5} \cdot y_{H_{2}}^{1.5}} = \exp\left(-\frac{\Delta g^{\circ}_{f,NH_{3}^{*}}(298.15\text{ K}; 0.15\text{ bar})}{RT}\right) \\
       Py_{H_{2}} = H_{H_{2}} \cdot x_{H_{2}} \\
       Py_{N_{2}} = H_{N_{2}} \cdot x_{N_{2}} \\
       Py_{NH_{3}} = H_{NH_{3}} \cdot x_{NH_{3}} \\
       Py_{H_{2}O} = P_{H_{2}O}^{\circ} \cdot x_{H_{2}O} 
   \end{cases}
\end{equation}

\section*{\color{bluepoli} TIPS}
\begin{itemize}
   \item The unknowns of our system are the lambda values
   \item To solve the problem, consider that the system of equations to be solved is (6); it contains the molar fractions of species in both gas and liquid phase, these can be easily written as functions of our problem's unknowns, namely the $\lambda$ values, as shown in system (1), and the relations reported in equations 2-3-4-5
   \item Reasonable initial guess values: $\lambda_{0}=[1, 0, 0, 0.5, 4.5]$
\end{itemize}
\end{document}
