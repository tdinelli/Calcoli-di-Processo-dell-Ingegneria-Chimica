\documentclass[oneside]{article}

% ---------------------------------------------
% Importing packages
% ---------------------------------------------

% Encoding and font
\usepackage[utf8]{inputenc}
\usepackage{tgcursor}
\usepackage{hyperref}
\usepackage{wrapfig}

% Different colors
\usepackage{xcolor}
\usepackage{color}
\definecolor{bluepoli}{cmyk}{0.4,0.1,0,0.4}

% Math
\usepackage{amsmath}
\usepackage{amsthm}

% Chemistry
\usepackage[version=4]{mhchem}

% Images
\usepackage{graphicx}
% \graphicspath{ {./Figures/} }

% Margins
\usepackage[a4paper, top=2cm, left=2.5cm, right=2.5cm, bottom=2cm]{geometry}

% Fancy header and footer
\usepackage{fancyhdr}
\pagestyle{fancy}
\fancyhf{}
\rhead{Calcoli di Processo dell'Ingegneria Chimica}
\lhead{Practical Session 10}
\rfoot{Page \thepage}
\lfoot{Academic Year 2024-2025}


\usepackage{amsthm}
\usepackage{tcolorbox}
\tcbuselibrary{most}
\tcolorboxenvironment{proof}{% `proof' from `amsthm'
   blanker,
   breakable,
   left=5mm,
   before skip=10pt,
   after skip=10pt,
   borderline west={1mm}{0pt}{bluepoli}
}

% ---------------------------------------------
% Title
% ---------------------------------------------

\title{Practical Session 10}
\author{Timoteo Dinelli\footnote{timoteo.dinelli@polimi.it}, Marco Mehl\footnote{marco.mehl@polimi.it}}
\date{12\textsuperscript{th} of December 2024}

% ---------------------------------------------
% Begin of the document
% ---------------------------------------------

\begin{document}
\maketitle

\section*{Exercise 1}
We want to study the start-up dynamics of a heating system. Thus, at the center of a
room, a spherical thermometer is placed and connected to the heating system through a
thermostat 
\[
D = 5 \, \text{mm}, \quad m = 10 \, \text{g}, \quad c_p = 500 \, \frac{\text{J}}{\text{kg K}}.
\]
When the temperature measured by the thermometer drops below $21.5^\circ \text{C}$, the
thermostat activates the heating. The room is empty, but the temperature recorded by the
thermometer does not represent the actual air temperature due to radiation exchange with
the walls. On a cold winter day, the walls are at a temperature of $T_p = 10^\circ
\text{C}$.

The heat transfer phenomena to consider are:
\begin{itemize}
   \item Convective heat exchange with the moving air inside the room, with $\nu = 1 \,
      \frac{\text{m}}{\text{s}}$.
   \item Radiative heat exchange with the walls, following the laws reported below:
\end{itemize}

\[
\dot{Q}_\text{rad} = \sigma A \left( T^4 - T_p^4 \right), \quad \sigma = 5.67 \times 10^{-8} \, \frac{\text{W}}{\text{m}^2 \, \text{K}^4},
\]

\[
\dot{Q}_\text{conv} = Nu \cdot \frac{k}{D} A \left( T_\text{air} - T \right), \quad Nu_D = 0.4 \, Re^{0.5} \, Pr^{1/3}.
\]

Assume the following properties for air:
\[
Pr = 0.7, \quad \mu = 1.81 \times 10^{-5} \, \text{Pa} \cdot \text{s}, \quad k = 0.025 \,
\frac{\text{W}}{\text{mK}}, \quad MW_\text{air} = 0.028 \, \frac{\text{kg}}{\text{mol}}.
\]

The energy balance for the thermometer is:
\[
m c_p \frac{dT}{dt} = - \dot{Q}_\text{rad} - \dot{Q}_\text{conv}.
\]

If at the initial time the thermometer measures $T_0 = 25^\circ \text{C}$, after 1 hour
of heating, what will be the measured temperature? Assume the wall temperature and the
air temperature in the room are constant, with $T_\text{air} = 20^\circ \text{C}$.

\section*{Exercise 2}
One of the first-aid remedies used to reduce the effects and pain in the case of trauma
is the application of cold compresses. This is often done using devices commonly called
"cold packs." A cold pack consists of a rectangular package containing water and a
capsule of NH\textsubscript{4}NO\textsubscript{3} (molar mass 80 g/mol), a salt whose
solution enthalpy is 26.2 kJ/mol. Typical dimensions for such packages are $10 \,
\text{cm} \times 15 \, \text{cm}$ with a weight of 200 g, of which 140 g is water.
Assuming an external temperature of $26^\circ \text{C}$ and a specific heat capacity of
the solution equal to $4.186 \, \text{J/g}^\circ\text{C}$, determine the temperature
reached by the cold pack after activation of the reaction ($T_0$) and its subsequent
evolution assuming that the package exchanges heat with the environment with a global
heat transfer coefficient $U = 40 \, \text{J/m}^2\text{ }^\circ\text{C}$. It is recalled
that the temperature derivative can be written as:
\[
\frac{dT}{dt} = \frac{(T_\text{amb} - T) \cdot (U \cdot A)}{c_p \cdot m}.
\]

Assuming that the cold pack loses its effectiveness when its temperature exceeds
$16^\circ \text{C}$, how many minutes does its effect last?

\section*{Exercise 3}
We are at the final stage of a rally race, the last time trial. The last two teams must
cover the final $20 \, \text{km}$, with both cars starting side by side and from rest.
The tension among the audience rises, and the bets begin! If
\[
v_A = 2\sqrt{t} + \cos(t + \pi/2) \, [\text{m/min}],
\]
and
\[
v_B = \sin(t) + 0.5t \, [\text{m/min}],
\]

Answer the following questions:
\begin{itemize}
   \item At what time does the first overtaking occur?
   \item After $1 \, \text{h}$ from the first overtaking, which car will be in the lead?
\end{itemize}

\end{document}
