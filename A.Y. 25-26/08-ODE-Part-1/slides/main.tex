\documentclass[aspectratio=169]{beamer}

% ==================================================================
% Define custom colors
\definecolor{primarycolor}{RGB}{25, 74, 166} % blue
\definecolor{accentcolor}{RGB}{65, 155, 232} % lighter blue
\definecolor{bluepoli}{RGB}{2, 30, 54}
\newcommand{\highlight}[2]{\colorbox{#1!9}{$#2$}}

% ==================================================================
% Apply these colors
\setbeamercolor{normal text}{fg=black, bg=white}
\setbeamercolor{alerted text}{fg=accentcolor}
\setbeamercolor{example text}{fg=accentcolor!80!black}
\setbeamercolor{progress bar}{fg=accentcolor, bg=accentcolor!20}
\setbeamercolor{frametitle}{bg=bluepoli, fg=bluepoli!80!black}
\setbeamercolor{title separator}{fg=bluepoli}

% ==================================================================
% Metropolis customization
\usetheme[sectionpage=none]{metropolis}
\setbeamercolor{background canvas}{bg=black!1.5}
\setbeamercolor{frametitle}{bg=black!1.5,fg=black}
\setbeamertemplate{sections/subsections in toc}[square]
\setbeamertemplate{footline}{
    \centerline{\textcolor{bluepoli}{\rule{0.95\paperwidth}{.3pt}}}
    \vskip2.5pt
    \hskip15pt \tiny Ordinary Differential Equations \hskip330pt \insertframenumber
    \vskip4pt
}

% ==================================================================
% Images
\usepackage{graphicx}

% ==================================================================
% Colors
\usepackage{color}
\usepackage[dvipsnames]{xcolor}
\usepackage{colortbl}

% ==================================================================
% Code rendering
\usepackage{minted}

% ==================================================================
% TIKZ
\usepackage{tikz}
\usetikzlibrary{positioning,tikzmark,backgrounds,arrows,shapes,calc}

% ==================================================================
% TITLE
\title{Ordinary Differential Equations\\Part 1}
\subtitle{Calcoli di Processo dell' Ingegneria Chimica}
\author[Dinelli]{\textbf{Timoteo~Dinelli}}
\institute{
   \inst{} Department of Chemistry, Materials and Chemical Engineering, ``Giulio Natta'', Politecnico di Milano.\\ \\
   \textbf{email}: timoteo.dinelli@polimi.it
}
\date{27\textsuperscript{th} of November 2025}

\begin{document}
{
    \setbeamertemplate{footline}{} 
    \begin{frame}{}
        \maketitle
        \begin{tikzpicture}[overlay, remember picture]
            \node[above left=3.6cm and 0.01cm of current page.south east]{\includegraphics[trim=1cm 1cm 5.5cm 1cm, clip=true, width=8cm]{figures/logo.pdf}};
        \end{tikzpicture}
    \end{frame}
}

% ==================================================================
% Slides
\begin{frame}{Ordinary Differential Equations}
    We will discuss different methods to approximate numerically the solution of the following Ordinary Differential Equation:
    
    \begin{equation*}
        \frac{dy}{dt} = f(t,y, \Theta)
    \end{equation*}

    Generally speaking we are going to solve what it is usually called an \textbf{IVP} (Initial Value Problems), a differential equation associated with a set of initial conditions.
    
    \begin{equation*}
        \begin{cases}
            \frac{dC_{A}}{dt} = - C_{A}\\
            C_{A}(t = t^{*}) = C_{A}^{*}
        \end{cases}
    \end{equation*}
\end{frame}

\begin{frame}{Forward Euler Method - Implementation}
    \textbf{Explicit scheme}: $y_{n+1} = y_{n} + hf(t_n, y_n)$
    
    \vspace{0.1cm}
    
    \textbf{Implementation steps}:
    \begin{enumerate}
        \item Start with initial condition: $y_0 = y(t_0)$
        \item Evaluate the derivative at current point: $f(t_n, y_n)$
        \item Update directly: $y_{n+1} = y_n + h \cdot f(t_n, y_n)$
        \item Advance time: $t_{n+1} = t_n + h$
    \end{enumerate}
    
    \vspace{0.1cm}
    
    \textbf{Practical example}: $\frac{dC_A}{dt} = -kC_A$, with $C_A(0) = 1$ mol/L, $k = 0.5$ s$^{-1}$, $h = 0.1$ s
    
    \begin{equation*}
        C_A(t_{n+1}) = C_A(t_n) + h \cdot (-k \cdot C_A(t_n))
    \end{equation*}
    
    \textbf{Key point}: Easy to implement, but watch out for stability with large $h$!
\end{frame}

\begin{frame}{Backward Euler Method - Implementation}
    \textbf{Implicit scheme}: $y_{n+1} = y_{n} + hf(t_{n+1}, y_{n+1})$
    
    \vspace{0.1cm}
    
    \textbf{Implementation steps}:
    \begin{enumerate}
        \item Start with initial condition: $y_0 = y(t_0)$
        \item Set up implicit equation: $y_{n+1} = y_n + h \cdot f(t_{n+1}, y_{n+1})$
        \item \textbf{Solve} for $y_{n+1}$ (requires root-finding or algebraic manipulation)
        \item Advance time: $t_{n+1} = t_n + h$
    \end{enumerate}
    
    \vspace{0.1cm}
    
    \textbf{Practical example}: $\frac{dC_A}{dt} = -kC_A$, with $C_A(0) = 1$ mol/L, $k = 0.5$ s$^{-1}$, $h = 0.1$ s
    
    \begin{equation*}
        C_A(t_{n+1}) = C_A(t_n) - h \cdot k \cdot C_A(t_{n+1}) \quad \Rightarrow \quad C_A(t_{n+1}) = \frac{C_A(t_n)}{1 + hk}
    \end{equation*}
    
    \textbf{Key point}: More stable, but requires solving (non)linear equations at each step!
\end{frame}

\begin{frame}{Methods Comparison}
    \begin{columns}
    \column{0.5\textwidth}
        \begin{figure}
            \centering
            \includegraphics[width=1.2\textwidth]{figures/expVSimp.png}
        \end{figure}

    \column{0.5\textwidth}
    \centering
        Forward Euler (\textbf{explicit}):
        \begin{equation*}
            y_{n+1} = y_{n} + hy_{n}^{'}
        \end{equation*}
        \centering
        Backward Euler (\textbf{implicit}):
        \begin{equation*}
            y_{n+1} = y_{n} + hy_{n+1}^{'}
        \end{equation*}
    \end{columns}
\end{frame}



{%
    \setbeamertemplate{footline}{}
    \begin{frame}[standout]
	   Exercises
    \end{frame}
}

\begin{frame}{Exercise 1: ODE Integration Methods}
    \textbf{Problem}: Solve $\frac{dx}{dt} = -kx$, with $x(0) = 1.0$
    
    \vspace{0.2cm}
    \small{Analytical solution: $x(t) = x_0 e^{-kt}$}
    
    \vspace{0.4cm}
    
    \textbf{Tasks}:
    \begin{enumerate}
        \item Implement \textbf{Forward Euler}: $x_{n+1} = x_n + hf(t_n, x_n)$
        \item Implement \textbf{Backward Euler}: $x_{n+1} = x_n + hf(t_{n+1}, x_{n+1})$
        \begin{itemize}
            \item For this linear ODE: $x_{n+1} = \frac{x_n}{1 + hk}$
        \end{itemize}
        \item Implement \textbf{Heun method with adaptive stepping}
        \begin{itemize}
            \item Use error estimation (1 full step vs 2 half steps)
            \item Adjust $h$ based on tolerance: $|error| < tol$
        \end{itemize}
    \end{enumerate}
\end{frame}

\begin{frame}{Exercise 1: Testing \& Analysis}
    \textbf{Test parameters}:
    \begin{itemize}
        \item $k = 1.0$, $x_0 = 1.0$, $t \in [0, 5]$
        \item Fixed step: $h = 0.1$ (Forward/Backward Euler)
        \item Adaptive: $h_{init} = 0.1$, $tol = 10^{-4}$ (Heun)
    \end{itemize}
    
    \vspace{0.1cm}
    
    \textbf{Stability study}: Test with $k = [0.1, 0.5, 1.0, 2.0, 5.0]$ and $h = 0.5$
    \begin{itemize}
        \item When does Forward Euler become unstable? (Hint: $hk > 2$)
        \item How does Backward Euler handle large $k$?
    \end{itemize}
\end{frame}

\begin{frame}{Exercise 2: Isothermal Batch Reactor}
    \textbf{Problem}: Design an isothermal batch reactor for the reaction:
    \begin{equation*}
        A \rightarrow B \quad k = 0.01 \: \text{s}^{-1}
    \end{equation*}
    
    \vspace{0.3cm}
    
    \textbf{Governing equations}:
    \begin{columns}[T]
        \column{0.48\textwidth}
        Species balances:
        \begin{equation*}
            \begin{cases}
                \frac{dC_{A}}{dt} = -k C_{A} \\
                \frac{dC_{B}}{dt} = k C_{A}
            \end{cases}
        \end{equation*}
        
        \column{0.48\textwidth}
        Conversion ($X_A$):
        \begin{equation*}
            X_A = \frac{C_A^0 - C_A}{C_A^0}
        \end{equation*}
        \begin{equation*}
            \frac{dX_A}{dt} = k(1-X_A)
        \end{equation*}
    \end{columns}
    
    \vspace{0.3cm}
    
    \small{Analytical solution: $X_A(t) = 1 - e^{-kt}$, \quad $C_A(t) = C_A^0 e^{-kt}$}
\end{frame}

\begin{frame}{Exercise 2: Implementation Tasks}
    \textbf{Initial conditions}: $C_A^0 = 1.0$ mol/L, $C_B^0 = 0$ mol/L, $X_A(0) = 0$
    
    \vspace{0.4cm}
    
    \textbf{Tasks}:
    \begin{enumerate}
        \item Solve the system using \texttt{ode45} (MATLAB built-in solver)
        \item Compare with your implemented methods (Forward/Backward Euler, Heun)
        \item Determine time to reach 90\% conversion ($X_A = 0.9$)
    \end{enumerate}
    
    \vspace{0.4cm}
    
    \textbf{Bonus}: Extend to a system of ODEs - solve for both $C_A$ and $C_B$ simultaneously and verify mass balance: $C_A(t) + C_B(t) = C_A^0$
\end{frame}

{
    \setbeamertemplate{footline}{}
    \begin{frame}[standout]
        Thank you for your attention!
    \end{frame}
}
\end{document}