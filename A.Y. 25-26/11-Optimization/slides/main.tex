\documentclass[aspectratio=169]{beamer}

% ==================================================================
% Define custom colors
\definecolor{primarycolor}{RGB}{25, 74, 166} % blue
\definecolor{accentcolor}{RGB}{65, 155, 232} % lighter blue
\definecolor{bluepoli}{RGB}{2, 30, 54}
\newcommand{\highlight}[2]{\colorbox{#1!9}{$#2$}}

% ==================================================================
% Apply these colors
\setbeamercolor{normal text}{fg=black, bg=white}
\setbeamercolor{alerted text}{fg=accentcolor}
\setbeamercolor{example text}{fg=accentcolor!80!black}
\setbeamercolor{progress bar}{fg=accentcolor, bg=accentcolor!20}
\setbeamercolor{frametitle}{bg=bluepoli, fg=bluepoli!80!black}
\setbeamercolor{title separator}{fg=bluepoli}

% ==================================================================
% Metropolis customization
\usetheme[sectionpage=none]{metropolis}
\setbeamercolor{background canvas}{bg=black!1.5}
\setbeamercolor{frametitle}{bg=black!1.5,fg=black}
\setbeamertemplate{sections/subsections in toc}[square]
\setbeamertemplate{footline}{
    \centerline{\textcolor{bluepoli}{\rule{0.95\paperwidth}{.3pt}}}
    \vskip2.5pt
    \hskip15pt \tiny Optimization \hskip380pt \insertframenumber
    \vskip4pt
}

% ==================================================================
% Images
\usepackage{graphicx}

% ==================================================================
% Colors
\usepackage{color}
\usepackage[dvipsnames]{xcolor}
\usepackage{colortbl}

% ==================================================================
% Code rendering
\usepackage{minted}

% ==================================================================
% TIKZ
\usepackage{tikz}
\usetikzlibrary{positioning,tikzmark,backgrounds,arrows,shapes,calc}

% ==================================================================
% TITLE
\title{Mono-dimensional optimization}
\subtitle{Calcoli di Processo dell' Ingegneria Chimica}
\author[Dinelli]{\textbf{Timoteo~Dinelli}}
\institute{
   \inst{} Department of Chemistry, Materials and Chemical Engineering, ``Giulio Natta'', Politecnico di Milano.\\ \\
   \textbf{email}: timoteo.dinelli@polimi.it
}
\date{11\textsuperscript{th} of December 2025}

\begin{document}
{
    \setbeamertemplate{footline}{} 
    \begin{frame}{}
        \maketitle
        \begin{tikzpicture}[overlay, remember picture]
            \node[above left=3.6cm and 0.01cm of current page.south east]{\includegraphics[trim=1cm 1cm 5.5cm 1cm, clip=true, width=8cm]{figures/logo.pdf}};
        \end{tikzpicture}
    \end{frame}
}

% ==================================================================
% Slides
{%
    \setbeamertemplate{footline}{}
    \begin{frame}[standout]
	   Exercises
    \end{frame}
}

\begin{frame}{Mono dimensional optimization}
   \footnotesize{
      In a one-dimensional optimization problem, it is necessary to identify the value of
      the variable $x$ (also known as the degree of freedom) that optimizes the objective
      function $f(x)$.

      The optimum of the problem can be represented by either the maximum or minimum of
      the objective function.

      \begin{equation*}
         \underset{x}{max} \left\{f(x)\right\} \quad or \quad \underset{x}{min}
         \left\{f(x)\right\}
      \end{equation*}

      The two problems are entirely equivalent. Indeed, it is sufficient to change the
      sign of the objective function to transform a minimization problem into a
      maximization problem and vice versa:

      \begin{align*}
         \underset{x}{max} \left\{f(x)\right\} &\equiv \underset{x}{min}
         \left\{-f(x)\right\} \\
         \underset{x}{min} \left\{g(x)\right\} &\equiv \underset{x}{max}
         \left\{-g(x)\right\}
      \end{align*}
   }
\end{frame}

\begin{frame}{}
   Using classical analysis knowledge incorrectly, one might think that to solve the
   problem:
   \begin{equation*}
      \underset{x}{min}\left\{f(x)\right\} \longrightarrow
      \frac{df(x)}{dx} = 0
   \end{equation*}
   Nothing could be more incorrect. It should be recalled that the aforementioned
   condition is necessary to identify an extremum point of a continuous and
   differentiable function. However, this condition does not cover specific cases such
   as the following examples:

   \begin{tikzpicture}[scale=0.8]
   \begin{scope}
      \draw[->] (0,0) -- (4,0) node[below] {$x$};
      \draw[->] (0,0) -- (0,3) node[left] {$f(x)$};
      \draw[blue, thick] plot[smooth] coordinates {
         (0,2) (0.5,1) (1,1.5) (1.5,1.2) (2,0.5) (3,1.2) (4,2)
      };
   \end{scope}
   \begin{scope}[xshift=6cm]
      \draw[->] (0,0) -- (4,0) node[below] {$x$};
      \draw[->] (0,0) -- (0,3) node[left] {$f(x)$};
      \draw[blue, thick] plot[smooth] coordinates {
         (0,2) (1,1.2) (2,0.5) (3,1.2) (4,2)
      };
   \end{scope}
   \begin{scope}[xshift=12cm]
      \draw[->] (0,0) -- (4,0) node[below] {$x$};
      \draw[->] (0,0) -- (0,3) node[left] {$f(x)$};
      % First segment (decreasing curve)
      \draw[blue, thick] plot[smooth] coordinates {
         (0.5,2.5) (1.5,1.5)
      };
      % Second segment (increasing curve)
      \draw[blue, thick] plot[smooth] coordinates {
         (2.5,1) (3.5,2)
      };
   \end{scope}
\end{tikzpicture}
\end{frame}

\begin{frame}
   \small{
      In one-dimensional function optimization, two main families of solution methods can be
      identified:

      \begin{itemize}
         \item[$\blacktriangleright$] \alert{Comparison Methods}: These methods rely
            exclusively on comparisons between objective function values.
         \item[$\blacktriangleright$] \alert{Approximation Methods}: These methods
            approximate the objective function with simpler functions and seek to find the
            optimal point of these approximating functions.
      \end{itemize}

      An alternative classification can also be made:

      \textcolor{blue}{Sequential Methods}: The evaluation of the objective function is
      conducted sequentially. Each evaluation enables the determination of subsequent
      actions to be taken.

      \textcolor{blue}{Parallel Methods}: These methods allow multiple function evaluations
      to be performed before deciding on the next strategy. Generally, these methods require
      significantly more function evaluations. They become particularly relevant when
      utilizing multiprocessor computing systems.
   }
\end{frame}

\begin{frame}{In Matlab}
   We are going to solve the following problem:
   \begin{columns}
   \column{0.5\textwidth}
      \begin{align*}
         \underset{x}{min} \: f(x) \: s.t. \:
         \begin{cases}
            c(x) \leq 0 \\
            ceq(x) = 0 \\
            A \cdot x \leq b \\
            Aeq \cdot x = beq \\
            lb \leq \: x \leq ub
         \end{cases}
      \end{align*}
   \column{0.5\textwidth}
      \begin{equation*}
         x = \textcolor{blue}{fmincon}(fun,x0,A,b)
      \end{equation*}
   \end{columns}
\end{frame}

{%
   \setbeamertemplate{footline}{}
   \begin{frame}[standout]
	   Exercises
   \end{frame}
}
\begin{frame}
   \begin{itemize}
      \item[$\blacktriangleright$] \textbf{Exercise 1}: The task is to determine the
         optimal dimensions of a cylindrical pressure vessel with a specified volume of
         $V = 100 m^3$. The following simplifying assumptions are made:
         \begin{itemize}
            \item Both ends (heads) are flat and closed.
            \item All vessel walls have constant thickness, $t$, and density, $\rho$.
            \item Manufacturing costs are equal for both heads and the lateral wall.
            \item Wall thickness, $t$, is independent of vessel pressure.
            \item There is no manufacturing waste.
         \end{itemize}
         The problem requires:
         \begin{itemize}
            \item Identification of possible objective function(s) to optimize
            \item Development of an analytical solution
            \item Proposal of an alternative numerical solution
            \item Determination of the optimal $L/D$ ratio (height to diameter ratio of the cylinder)
         \end{itemize}

      % \item[$\blacktriangleright$]
      % \item[$\blacktriangleright$]
      % \item[$\blacktriangleright$]
      % \item[$\blacktriangleright$]
   \end{itemize}
\end{frame}

\begin{frame}
   \begin{itemize}
      \item[$\blacktriangleright$] \textbf{Exercise 2}: By removing the simplifying
         assumptions expressed in Exercise 1, we arrive at a more complex objective
         function that takes into account the following considerations:
         \begin{itemize}
            \item The end caps are ellipsoidal, thus having a larger surface area
               compared to the flat surface assumption
            \item The end caps are more expensive to manufacture than the lateral surface
            \item The sheet metal thickness is a function of the vessel diameter.
         \end{itemize}
         In this case, the objective function to be optimized is:
         \begin{equation*}
            f_{obj} = 0.0432V + 0.5 \frac{V}{D} + 0.3041D^2 + 0.0263D^3
         \end{equation*}
         Determine the optimal diameter $D$ and the height $L$  knowing that:
         $V=\frac{\pi D^2}{4}\left(L+\frac{D}{2}\right)$.
   \end{itemize}
\end{frame}

\begin{frame}
   \begin{itemize}
      \item[$\blacktriangleright$] \textbf{Exercise 3}: In radiation, the spectral power
         emitted by a black body can be expressed using the following relation:
         \begin{equation*}
            E_b \left( \lambda , T \right) = \frac{C_1}{\lambda ^5
            exp\left(\frac{C_2}{\lambda T}\right)^{-1}}
         \end{equation*}
         Where:
         \begin{itemize}
            \item[ ] $C_1 = 3.742 \times 10^8 \: W \cdot \mu m^4 / m^2$
            \item[ ] $C_2 = 1.439 \times 10^4 \: \mu m \cdot K$
         \end{itemize}

         Determine the peak power emission point of a black body at a temperature of $800
         \: K$ by plotting the behavior of $E_b$ versus electromagnetic radiation
         wavelength on a log-log graph. Compare the wavelength value found through
         numerical procedure with that obtained from Wien's Law:

         \begin{equation*}
            \lambda_{max}T = C_{3} \quad C_3 = 2897.8 \: \mu m \cdot K
         \end{equation*}
   \end{itemize}
\end{frame}

\begin{frame}
   \scriptsize{
      \begin{itemize}
         \item[$\blacktriangleright$] \textbf{Exercise 4}: In a shock tube, autoignition
            delay times of a fuel are measured at a temperature of $1000 \: K$ under varying
            pressure conditions as reported in the following table:
            \begin{table}[]
               \begin{tabular}{c  c}
                  P [atm] & tau [ms]\\ \hline
                  3  & 0.75 \\
                  10 & 0.20 \\
                  20 & 0.10 \\
                  30 & 0.55 \\
                  40 & 0.04 \\
                  60 & 0.03 \\ \hline
               \end{tabular}
            \end{table}
            A model is to be formulated to estimate the autoignition delay time as a
            function of pressure, using the data point at $10 \: atm$ as a reference value,
            according to the formula:
            \begin{equation*}
               \tau = \tau_0 \left(\frac{P}{P_0}\right)^\beta \quad  \tau_0 = \tau (10\,
               atm) \quad P_0 = 10 \, atm
            \end{equation*}
            Compute the parameter $\beta$ using the method of the sum of the squared errors,
            in order to solve the problem:
            \begin{itemize}
               \item The definition of a model function that takes pressure and $\beta$ as
                  inputs and returns the value of $\tau$.
               \item The definition of an objective function that computes the sum of
                  squared residuals $f(x) = \sum_{i} \left(\tau_i -
                  \tau_{mod_{i}}\right)^2$, taking $\beta$ as input.
            \end{itemize}
      \end{itemize}
   }
\end{frame}

{
    \setbeamertemplate{footline}{}
    \begin{frame}[standout]
        Thank you for your attention!
    \end{frame}
}
\end{document}
